%  LaTeX support: latex@mdpi.com
%  In case you need support, please attach any log files that you could have, and specify the details of your LaTeX setup (which operating system and LaTeX version / tools you are using).

%=================================================================

% LaTeX Class File and Rendering Mode (choose one)
% You will need to save the "mdpi.cls" and "mdpi.bst" files into the same folder as this template file.

%=================================================================

\documentclass[entropy,article,accept,oneauthor,pdftex,10pt,a4paper]{mdpi}
%--------------------
% Class Options:
%--------------------
% journal
%----------
% Choose between the following MDPI journals:
% actuators, administrativesciences, aerospace, agriculture, agronomy, algorithms, animals, antibiotics, antibodies, antioxidants, appliedsciences, arts, atmosphere, atoms, axioms, batteries, behavioralsciences, beverages, bioengineering, biology, biomedicines, biomimetics, biomolecules, biosensors, brainsciences, buildings, cancers, catalysts, cells, challenges, chemosensors, children, chromatography, climate, coatings, computation, computers, cosmetics, crystals, data, dentistryjournal, diagnostics, diseases, diversity, econometrics, economies, education, electronics, energies, entropy, environments, epigenomes, fermentation, fibers, foods, forests, futureinternet, galaxies, games, gels, genealogy, genes, geosciences, geriatrics, healthcare, horticulturae, humanities, hydrology, informatics, information, inorganics, insects, ijerph, ijfs, ijms, ijns, ijgi, jcdd, jcm, jdb, jfb, jfmk, jimaging, jof, joi, jlpea, jmse, jpm, jrfm, jsan, land, languages, laws, life, lubricants, machines, marinedrugs, materials, mathematics, medicalsciences, membranes, metabolites, metals, microarrays, micromachines, microorganisms, minerals, molbank, molecules, nanomaterials, ncrna, nutrients, pathogens, pharmaceuticals, pharmaceutics, pharmacy, philosophies, photonics, plants, polymers, processes, proteomes, publications, recycling, religions, remotesensing, resources, risks, robotics, safety, sensors, sinusitis, socialsciences, societies, sports, standards, sustainability, symmetry, systems, technologies, toxics, toxins, universe, vaccines, veterinarysciences, viruses, water
%---------
% article
%---------
% The default type of manuscript is article, but could be replaced by using one of the class options:
% article, review, communication, commentary, bookreview, correction, addendum, editorial, changes, supfile, casereport, comment, conceptpaper, conferencereport, meetingreport, discussion, essay, letter, newbookreceived, opinion, projectreport, reply, retraction, shortnote, technicalnote, creative, datadescriptor (for journal Data), briefreport, hypothesis, interestingimage
%----------
% submit
%----------
% The class option "submit" will be changed to "accept" by the Editorial Office when the paper is accepted. This will only make changes to the frontpage (e.g. the logo of the journal will get visible), the headings, and the copyright information. Journal info and pagination for accepted papers will also be assigned by the Editorial Office.
% Please insert a blank line is before and after all equation and eqnarray environments to ensure proper line numbering when option submit is chosen
%------------------
% moreauthors
%------------------
% If there is only one author the class option oneauthor should be used. Otherwise use the class option moreauthors.
%---------
% pdftex
%---------
% The option "pdftex" is for use with pdfLaTeX only. If eps figure are used, use the optioin "dvipdfm", with LaTeX and dvi2pdf only.

%=================================================================
\setcounter{page}{1}
\lastpage{x}
\doinum{}       %\doinum{10.3390/------}
\pubvolume{}    % \pubvolume{xx}
\pubyear{2017}
%\externaleditor{Academic Editor: xx}
\history{In preparation}%\history{Received: xx / Accepted: xx / Published: xx}

\usepackage{dsfont}
\usepackage[utf8]{inputenc}
\usepackage{algorithm, mathtools}
\usepackage[noend]{algpseudocode}
\usepackage{amsmath}
\usepackage{amssymb}
\usepackage{natbib}
\usepackage{enumerate}
\usepackage{microtype}

\DeclareUnicodeCharacter{00A0}{ }

\definecolor{orange}{rgb}{1, 0.5, 0}
\definecolor{green}{rgb}{0, 0.5, 0}

\renewcommand{\d}{\boldsymbol{d}}
\newcommand{\todo}{\color{orange} \bf}
\newcommand{\query}{\color{green} \bf}
\newcommand{\x}{\boldsymbol{\theta}}
\newcommand{\bphi}{\boldsymbol{\phi}}
\newcommand{\boldeta}{\boldsymbol{\eta}}
\newcommand{\n}{\boldsymbol{\eta}}
\newcommand{\depth}{(\textnormal{depth})}
\newcommand{\xref}{\x_{\rm ref}}

\newcommand{\apj}{The Astrophysical Journal}
\newcommand{\mnras}{Monthly Notices of the Royal Astronomical Society}

%------------------------------------------------------------------
% The following line should be uncommented if the LaTeX file is uploaded to arXiv.org
%\pdfoutput=1

%=================================================================

% Add packages and commands to include here
% The amsmath, amsthm, amssymb, hyperref, caption, float and color packages are loaded by the MDPI class.
%\usepackage{graphicx}
%\usepackage{subfigure,psfig}

%=================================================================
%% Please use the following mathematics environments:
% \theoremstyle{mdpi}
% \newcounter{thm}
% \setcounter{thm}{0}
% \newcounter{ex}
% \setcounter{ex}{0}
% \newcounter{re}
% \setcounter{re}{0}
%
% \newtheorem{Theorem}[thm]{Theorem}
% \newtheorem{Lemma}[thm]{Lemma}
% \newtheorem{Corollary}[thm]{Corollary}
% \newtheorem{Proposition}[thm]{Proposition}
%
% \theoremstyle{mdpidefinition}
% \newtheorem{Characterization}[thm]{Characterization}
% \newtheorem{Property}[thm]{Property}
% \newtheorem{Problem}[thm]{Problem}
% \newtheorem{Example}[ex]{Example}
% \newtheorem{ExamplesandDefinitions}[ex]{Examples and Definitions}
% \newtheorem{Remark}[re]{Remark}
% \newtheorem{Definition}[thm]{Definition}
%% For proofs, please use the proof environment (the amsthm package is loaded by the MDPI class).

%=================================================================

% Full title of the paper (Capitalized)
\Title{Computing Entropies With Nested Sampling}

% Authors (Add full first names)
\Author{Brendon J. Brewer$^{1,}$*}

% Affiliations / Addresses (Add [1] after \address if there is only one affiliation.)
\address{
$^{1}$ Department of Statistics, The University of Auckland, Private Bag 92019,
Auckland 1142, New Zealand}

%\contributed{$^\dagger$ These authors contributed equally to this work.}

% Contact information of the corresponding author (Add [2] after \corres if there are more than one corresponding author.)
\corres{{\tt bj.brewer@auckland.ac.nz}}

% Abstract (Do not use inserted blank lines, i.e. \\)
\abstract{The Shannon entropy, and related quantities such as mutual
information, can be used to quantify uncertainty and relevance.
However, in practice, it can be difficult to compute these quantities
for arbitrary probability distributions, particularly if the probability
mass functions or densities cannot be evaluated. This paper introduces a
computational approach, based on Nested Sampling, to evaluate entropies of
probability distributions that can only be sampled. I demonstrate the method
on a simple gaussian example where the key quantities are known analytically,
and an experimental design example about scheduling observations in order
to measure the period of an oscillating signal.}

% Keywords: add 3 to 10 keywords
\keyword{information theory; entropy; mutual information; monte carlo;
nested sampling; bayesian inference}

% The fields PACS, MSC, and JEL may be left empty or commented out if not applicable
%\PACS{}
%\MSC{}
%\JEL{}

% If this is an expanded version of a conference paper, please cite it here: enter the full citation of your conference paper, and add $^\dagger$ in the end of the title of this article.
%\conference{}

%%%%%%%%%%%%%%%%%%%%%%%%%%%%%%%%%%%%%%%%%%
% For journal Data:

%\dataset{DOI number or link to the deposited data set in cases where the data set is published or set to be published separately. If the data set is submitted and will be published as a supplement to this paper in the journal Data, this field will be filled by the editors of the journal. In this case, please make sure to submit the data set as a supplement when entering your manuscript into our manuscript editorial system.}
%\datasetlicense{license under which the data set is made available (CC0, CC-BY, CC-BY-SA, CC-BY-NC, etc.)}

%%%%%%%%%%%%%%%%%%%%%%%%%%%%%%%%%%%%%%%%%%

\begin{document}

%%%%%%%%%%%%%%%%%%%%%%%%%%%%%%%%%%%%%%%%%%

\section{Introduction}

If an unknown quantity $x$ has a discrete probability distribution $p(x)$,
the Shannon entropy \citep{shannon} is defined as
\begin{align}
H(x) &= -\sum_{x} p(x) \log p(x)
\end{align}
where the sum is over all of the possible values of $x$ under consideration.
The entropy quantifies the degree to which the issue
``what is the value of $x$, precisely?'' remains unresolved
\citep{knuth_questions}. For example,
if there is only one possible value which has probability 1, $H(x)$ is
zero. Conventionally, $0 \times \log 0$ is defined to be equal
to $\lim_{x \to 0^+} (x\log x)$, which is zero; i.e., `possibilities' with
probability zero do not contribute to the sum.
If there are $N$ possibilities with equal probabilities $1/N$ each,
then the entropy is $H(x) = \log(N)$.
If the space of possible $x$-values is continuous so that $p(x)$ is a
probability density function, the differential entropy
\begin{align}
H(x) &= -\int p(x) \log p(x) \, dx
\end{align}
quantifies uncertainty by generalising the log-volume
of the plausible region, when volume is defined with respect to $dx$.

Entropies, and related quantities, tend to be analytically available only for
a few families of probability distributions. On the numerical side, if
$\log p(x)$ can be evaluated for any $x$, then simple Monte Carlo will suffice
for approximating $H(x)$. On the other hand, if $p(x)$ can only be
{\em sampled} but not evaluated (for example, if $p(x)$ is a marginal
distribution), then this will not work. Kernel density estimation or the like
\citep[e.g.][]{JMLR:v15:szabo14a}
may be effective in this case, but is unlikely to generalise well to high
dimensions.

This paper introduces a computational approach to evaluating the entropy 
$H(x)$ of probability distributions that can only be sampled
(using Markov Chain Monte Carlo).

\subsection{Notation and conventions}

Throughout this paper, I use the compact
`overloaded' notation for probability distributions favoured by many
Bayesian writers \citep{jaynes2003probability, mackay2003information},
writing $p(x)$ for either a probability mass function
or a probability density function, instead
of the more cumbersome $P(X=x)$ or $f_X(x)$.
In the compact notation, there is no distinction between the
`random variable' itself ($X$) and a similarly-named dummy variable ($x$).
Probability distributions are implicitly conditional on some prior
information, which is omitted from the notation unless necessary.
All logarithms are written $\log$ and any base can be used, unless otherwise
specified (for example by writing $\ln$ or $\log_{10}$). Any numerical
values given for the value of specific entropies are in {\em nats}, i.e.,
the natural logarithm was used.

Even though the entropy is written as $H(x)$, it is imperative that we
remember it is not a property of the value of $x$ itself, but a property
of the probability distribution used to describe a state of ignorance about $x$.
Throughout this paper, $H(x)$ is used as notation for both Shannon entropies
(in discrete cases) and differential entropies (in continuous cases). Which
one it is should be clear from the context of the problem at hand.

\section{Entropies in Bayesian inference}

Bayesian inference is the use of probability theory to
describe uncertainty, often about unknown quantities
(`parameters') $\x$. Some data $\d$, initially unknown but
thought to be relevant to $\x$, is obtained.
The prior information leads the
user to specify a prior distribution $p(\x)$ for the unknown parameters,
along with a conditional distribution $p(\d | \x)$ describing prior knowledge
of how the data is related to the parameters
(i.e., if the parameters where known, what data would likely be observed?).
By the product rule, this yields a {\em joint prior}
\begin{align}
p(\x, \d) &= p(\x)p(\d | \x)
\end{align}
which is the starting point for
Bayesian inference \citep{caticha2008lectures, caticha2006updating}.
In practical applications, the following operations
are usually feasible and have a low computational cost:
\begin{enumerate}
  \item Points can be generated from the prior $p(\x)$;
  \item Simulated datasets can be generated from $p(\d | \x)$ for any
        given value of $\x$;
  \item The likelihood, $p(\d | \x)$, can be evaluated cheaply for any
        $\d$ and $\x$. Usually it is the log-likelihood that is actually
        implemented, for numerical reasons.
\end{enumerate}
Throughout this paper I assume these operations are available and inexpensive.

\subsection{The relevance of data}

The entropy of the prior $p(\x)$ describes the degree to which the question
``what is the value of $\x$, precisely?'' remains unanswered, while the
entropy of the joint prior $p(\x, \d)$
describes the degree to which the question
``what is the value of the pair $(\x, \d)$, precisely?'' remains unanswered.
The degree to which the question ``what is the value of $\x$?'' would remain
unresolved if $\d$ were resolved is given by the
conditional entropy
\begin{align}
H(\x | \d) &= - \sum_{\d} p(\d) \sum_{\x} p(\x | \d) \log p(\x | \d)
\end{align}
which is the expected value of the entropy of the posterior, averaged over
all possible datasets which might be observed. 

Suppose we wanted to compute $H(\x | \d)$, perhaps to compare it to
$H(\x)$ and quantify how much might be learned about $\x$.
This would be difficult because the expression
for the posterior distribution
\begin{align}
p(\x | \d) &= \frac{p(\x)p(\d | \x)}{p(\d)}
\end{align}
contains the marginal likelihood integral:
\begin{align}
p(\d) &= \int p(\x) p(\d | \x) \, d\x
\end{align}
also known as the `evidence', which tends to be computable but costly.

It is important to distinguish between $H(\x | \d)$ and the
entropy of $\x$ given a particular value of $\d$, which might be written
$H(\x | \d=\d_0)$.
The former measures the degree to which one question answers
another {\em ex ante}, and is a function of two questions.
The latter measures the degree to which a {\em statement} answered a question
{\em ex post}, and is a function of a question and a statement.

\subsection{Mutual information}

The mutual information is another way of describing the relevance of the
data to the parameters. Its definition, and relation to other quantities, is
\begin{align}
I(\x; \d) &= \sum_{\x} \sum_{\d} p(\x, \d)
                       \log\left[\frac{p(\x, \d)}{p(\x)p(\d)}\right]\\
           &= H(\x) + H(\d) - H(\x, \d)\\
           &= H(\d) - H(\d | \x)\\
           &= H(\x) - H(\x | \d).
\end{align}
The mutual information can also be written as the expected value
(with respect to the prior over datasets $p(\d)$) of the Kullback-Leibler
divergence from prior to posterior:
\begin{align}
I(\x ; \d) &= \sum_{\d} p(d) D_{\rm KL}\big(p(\x|\d) \, || \, p(\x)\big).
\end{align}
In terms of the prior, likelihood, and evidence,
it is
\begin{align}
I(\x; \d) &= \sum_{\x} \sum_{\d} p(\x)p(\d | \x)
              \left[\log p(\d | \x) - \log p(\d)\right],
\end{align}
i.e., the mutual information is the prior expected value of the
log likelihood minus the log evidence.
As with the conditional entropy, we see that the computational
difficulty appears in the form of the log evidence,
$\log p(\d) = \log \sum_{\x} p(\x)p(\d | \x)$.

For experimental design purposes, if we want to maximise the expected amount
of information obtained from data, maximising either $I(\x; \d)$ or
$H(\x | \d)$ will produce the same result because the prior $p(\x)$ does not
vary with the experimental design. Reference priors
\citep{bernardo2005reference} also maximise $I(\x; \d)$ but vary the prior
in the maximisation process while keeping the experimental design fixed.

\section{Nested Sampling}

Nested Sampling \citep{skilling2006nested} is an algorithm whose aim is
to calculate the evidence
\begin{align}
Z &= \int \pi(\x) L(\x) \, d\x
\end{align}
where $\x$ is the unknown parameter(s), $\pi$ is the prior distribution,
and $L$ is the likelihood function. While $Z$ is a simple expectation with
respect to $\pi$, the
implied distribution of $L$-values tends to be very heavy-tailed, which is
why simple Monte Carlo does not work. Equivalently, the integral is dominated
by very small regions where $L(\x)$ is high.

To overcome this, NS evolves a population of $N$ particles in the parameter
space.
The particles are initially initialised from the prior $\pi(\x)$, and the
particle with the lowest likelihood, $L^*_1$, is found and its details are
recorded as output. This worst particle is then discarded
and replaced by a new particle
drawn from the prior $\pi$ but subject to the constraint that its likelihood
must be above $L^*_1$ (this is usually achieved by cloning a surviving particle
and evolving it with MCMC). This process is repeated, resulting in an
increasing sequence of likelihoods
\begin{align}
L^*_1, L^*_2, L^*_3,..., \label{eqn:likelihood_sequence}
\end{align}

Defining $X(\ell)$ as the amount of prior mass with likelihood greater than
some threshold $\ell$,
\begin{align}
X(\ell) &= \int \pi(\x) \mathds{1}\left(L(\x) > \ell\right) \, d\x,
\end{align}
we can assign corresponding $X$-values to each discarded point in the sequence,
and transform $Z$ to a one-dimensional integral.
\citet{skilling2006nested}'s key insight was that we can estimate the corresponding
$X$-values of the discarded points from the nature of the
algorithm. Specifically, each iteration reduces the prior mass by approximately
a factor $e^{-1/N}$. More accurately, the conditional distribution of
$X_{i}$ given $X_{i-1}$ (and the definition of the algorithm)
is obtainable by letting
$t_i \sim \textnormal{Beta}(N,1)$ and setting $X_{i+1} := t_iX_{i-1}$.
The $t_i$ variables represent the proportion of the remaining prior mass
that is {\em retained} at each iteration, after imposing the constraint of the
latest $L^*$ value.

\subsection{The sequence of $X$ values}

Consider $1-t_i$, the fraction of the remaining prior mass that is {\em removed}
at each iteration.
The distribution for the compression factors $t_i \sim $Beta$(N,1)$
corresponds to an exponential distribution for $\ln(1-t_i)$:
\begin{align}
\ln (1 - t_i) &\sim \textnormal{Exponential}(N).
\end{align}

Therefore,
the sequence of $-\ln X$ values produced by a Nested Sampling run can be
considered as a Poisson process with rate equal to $N$, the number of
particles. This is why separate NS runs can be simply merged
\citep{skilling2006nested}.
\citet{Walter2015} showed how this view of the NS sequence of points
can be used to construct a
version of NS that produces unbiased (in the frequentist sense)
estimates of the evidence.
However, it can also be used to construct an unbiased estimator of
a log-probability, which is more relevant to information theoretic
quantities discussed in the present paper.

Consider a particular likelihood value $\ell$ whose corresponding $X$-value
we would like to know. Since the $-\ln X_i$ values have the same distribution
as the arrival times of a Poisson process with rate $N$, the probability
distribution for the number of points in an interval of length $w$ is
Poisson with expected value $Nw$. Therefore the expected number of
NS discarded points with likelihood below $\ell$ is $-N\ln X(\ell)$:
\begin{align}
\big< n(L(\x_i) \leq \ell) \big> &= -N\ln X(\ell).
\end{align}
We can therefore take $n(L(\x_i) \leq \ell)/N$, the number of points in the
sequence with likelihood below $\ell$ divided by the number of NS particles,
as an unbiased estimator of $-\ln X(\ell)$.

\section{The algorithm}

The above insight, that the number of NS discarded points with likelihood
below $\ell$ has expected value $-N\ln X(\ell)$, is the basis of the algorithm.
If we wish to measure a log-probability $-\ln X(\ell)$, we can use NS to do it.
$\pi(\x)$ and $L(\x)$ need not be the prior and likelihood respectively, but
can be any probability distribution (over any space) and any function whose
expected value is needed for a particular application.

See Algorithm~\ref{alg:algorithm} for a step-by-step description of the
algorithm. The algorithm is written in terms of an unknown quantity $\x$
whose entropy is required. In applications, $\x$ may be parameters,
a subset of the parameters, data, some function of the data,
parameters and data jointly, or whatever.

The idea is to generate a reference particle $\xref$
from the distribution $p(\x)$
whose entropy is required. Then, a Nested Sampling run evolves a set of
particles $\{\x\}_{i=1}^N$, initially representing $p(\x)$,
towards $\xref$ in order to measure the log-probability near
$\xref$ (see Figure~\ref{fig:fig:algorithm}).
Nearness is defined using a distance function
$d(\x; \xref)$, and the number of NS iterations taken to reduce
the distance to below some threshold $L$ provides an unbiased estimate of
\begin{align}
   \depth = -\log \left[ P(d(\x; \xref) < L) \right]
\end{align}
which I call the `depth'. E.g., if $N=10$ Nested Sampling particles are used,
and it takes 100 iterations to reduce the distance to below $L$, then the
depth is 10 nats.

If we actually want the differential entropy
\begin{align}
H(\x) &= -\int p(\x) \log p(\x) \, d\x
\end{align}
we can use the fact that density equals mass divided
by volume. Assume $L$ is small, so that
\begin{align}
P(d(\xref, \x) < \epsilon | \x)
    &\approx
    p(\x) \int_{d(\xref, \x) < L} \, d\xref\\
    &= f(\x) \, V
\end{align}
where $V$ is the volume of the region where $d(\xref, \x) < L$.
Then $H(\x) = \left< \depth \right>_{p(\x)} + \log V(L)$.

\begin{algorithm}
\begin{algorithmic}
\State {\em Set the numerical parameters:}
\State $N \in \{1, 2, 3, ... \}$
            \Comment{the number of Nested Sampling particles to use}
\State $L \geq 0$
            \Comment{the tolerance} \\
\hrulefill
\State $\widehat{\mathbf{h}} \leftarrow [\,]$
            \Comment{Initialise an empty list of results}

\While{more iterations desired}
    \State $k \leftarrow 0$
            \Comment{Initialise counter}
    \State Generate $\xref$ from $p(\x)$
            \Comment{Generate a reference point}
\color{green}
    \State Generate $\left\{\x_i\right\}_{i=1}^N$ from $p(\x)$
            \Comment{Generate initial NS particles}
    \State Calculate $d_i \leftarrow d(\x_i; \xref)$ for all $i$
            \Comment{Calculate distance of each particle from the reference
                     point}
    \State $i^* \leftarrow \textnormal{argmin}(\lambda i \rightarrow d_i)$
            \Comment{Find the worst particle (greatest distance)}
    \State $d_{\rm max} \leftarrow d_{i^*}$
            \Comment{Find the greatest distance}
    \While{$d_{\rm max} > L$}
        \State Replace $\x_{i^*}$ with $\x_{\rm new}$ from
               $p\left(\x | (d(\x) < d_{\rm max})\right)$
            \Comment{Replace worst particle}
        \State Calculate $d_{i^*} \leftarrow d(\x_{\rm new}; \xref)$
            \Comment{Calculate distance of new particle from reference point}
        \State $i^* \leftarrow \textnormal{argmin}(\lambda i \rightarrow d_i)$
                \Comment{Find the worst particle (greatest distance)}
        \State $d_{\rm max} \leftarrow d_{i^*}$
                \Comment{Find the greatest distance}
        \State $k \leftarrow k+1$
            \Comment{Increment counter $k$}
    \EndWhile
\color{black}
    \State $\widehat{\mathbf{h}} \leftarrow \widehat{\mathbf{h}} + [k/N]$
            \Comment{Append latest estimate to results}
\EndWhile
\State $\widehat{h_{\rm final}} \leftarrow \frac{1}{\rm num\_iterations} \sum \widehat{\mathbf{h}}$
            \Comment{Average results}
\end{algorithmic}
\caption{The algorithm which estimates the `depth',\\
           \quad\quad$-\int p(\x) \int p(\xref)
            \log \left[ P(d(\xref, \x) < L | \x) \right]
                        \, d\xref \, d\x$,\\ minus the expected value of the
        log-probability of a small region near $\x$, which can be converted
        to an estimate of an entropy or differential entropy.
        The part highlighted in green is standard Nested Sampling
        with quasi-prior $p(\x)$ and quasi-likelihood given by minus a
        distance function $d(\xref, \x)$.
        \label{alg:algorithm}}
\end{algorithm}

\begin{figure}[!ht]
\centering
\includegraphics[width=0.5\linewidth]{figures/algorithm.pdf}
\caption{To evaluate the log-probability (or density) of the blue
probability distribution at the red point, Nested Sampling can be used,
with the blue distribution plays the role of the ``prior'' in NS, and the
Euclidian distance from the red point (illustrated with red contours)
is the ``likelihood''.\label{fig:algorithm}}
\end{figure}

If the distance function $d(\xref, \x)$ is chosen to be Euclidean,
the constraint $d(\xref, \x) < L$ corresponds to a ball in the space
of possible $\xref$ values.
The log-volume of a ball of radius $L$ in $n$ dimensions is
\begin{align}
\log V(\epsilon; n) &= \frac{n}{2}\log \pi
                        - \log \Gamma\left(\frac{n}{2} + 1\right)
                        + n \log L,
\end{align}

If the threshold $L$ is not small, we might think we are calculating the entropy associated
with the question `what is $x$, to within a tolerance of $\pm L$?'.
This is not quite correct. See the Appendix for a discussion of this
type of question.

In the following sections I demonstrate the algorithm on two problems.

\section{Example 1: Entropy of a prior for the data}

This example demonstrates how the algorithm may be used to determine the
entropy for a quantity whose distribution can only be sampled, in this case,
a prior distribution over datasets.

Consider the basic statistics problem of inferring a quantity $\mu$ from
100 observations $\d = \{d_1, ..., d_{100}\}$ whose
probability distribution (conditional on $\mu$) is
\begin{align}
p(d_i | \mu) &\sim \textnormal{Normal}(\mu, 1).
\end{align}
If the prior for $\mu$ is Normal$(0, 10^2)$, then the posterior is
\begin{align}
\mu | \d &\sim \textnormal{Normal}\left(
                                       \frac{1}{100}\sum_{i=1}^{100} d_i,
                                       \left[\frac{10}{\sqrt{10001}}\right]^2
                                       \right).
\end{align}
The true value of the differential entropy $H(\d)$ is available analytically
in this case, and its value is $146.499$ nats. However, in most situations, the
entropy of a marginal distribution is not available in closed form. 

I ran the algorithm and computed the depth using a tolerance of
$L=10^{-3}\sqrt{100}$, so that the RMS difference between
points in the two datasets was about $10^{-3}$.
From the algorithm, the average depth was
\begin{align}
\depth_{\d} &= 698.25 \pm 0.34.
\end{align}
The log-volume of a 100-dimensional ball of radius
$L=10^{-3}\sqrt{100}$ is
$-551.76$. Therefore, the differential entropy is estimated to be
\begin{align}
H(\d) &\approx 146.49 \pm 0.34.
\end{align}
which is very close to the true value (suspiciously close, but this
was just a fluke).

\section{Example 2: Measuring the period of an oscillating signal}

In physics and astronomy, it is common to measure an oscillating signal
at a set of times $\{t_1, ..., t_n\}$, in order to infer the amplitude,
period, and phase of the signal.
Here, I demonstrate the algorithm on a toy version
of Bayesian experimental design: at what times should the signal be
observed in order to obtain as much information as possible about the
period? To answer this, we need to be able to calculate the mutual
information between the unknown period and the data.

As the algorithm lets us calculate the entropy of any distribution which can
be sampled by MCMC, there are several options. The mutual information
can be written in various ways, such as the following.
\begin{align}
I(\x; \d) &= H(\x) + H(\d) - H(\x, \d) \\
I(\x; \d) &= H(\d) - H(\d|\x) \\
I(\x; \d) &= H(\x) - H(\x|\d) \\
I(\x; \d) &= \int p(\d) \, D_{\rm KL}\left(p(\x|\d) \,||\, p(\x)\right) \, d\d.
\end{align}
In a Bayesian problem where $\x$ is an unknown parameter and $\d$ is data, the
first and second formulas would be costly to compute, because the
high dimensional probability distribution for the dataset would require a
large number of NS iterations to compute $H(\d)$ and $H(\x, \d)$.
The fourth involves the Kullback-Leibler divergence from the prior to the
posterior, averaged over all possible datasets. This is straightforward to
approximate with standard Nested Sampling, but the estimate for a given dataset
may be biased. This method also would not work if we want $I(\x; \d)$ for
a single parameter or a subset of them, rather than for all of the parameters.
Therefore, I adopted the third formula as the best way to compute $I(\x; \d)$.
In the example, $H(\x)$ is available analytically, and I use the algorithm to
obtain $H(\x | \d)$ as follows:
\begin{align}
H(\x | \d) &= \int p(\d') H(\x | \d=\d') \, d\d'.
\end{align}
This is an expected value over datasets, of the entropy of the posterior given
each dataset. To do the Nested Sampling to estimate $H(\x | \d=\d')$, I generate
initial particles from the posterior by doing standard MCMC (targeting the
posterior) using the $\x$-value that produced the simulated dataset $\d'$ as
the initial condition for the MCMC.

\subsection{Assumptions}

I consider two possible observing strategies, both involving
$n=101$ measurements. The `even' strategy has observations at times
\begin{align}
t_i = \frac{i-1}{n-1}
\end{align}
for $i \in \{1, 2, ..., 101\}$, that is, the observation times are
$0, 0.01, 0.02, ..., 0.99, 1$ ---
evenly spaced from $t=0$ to $t=1$, including the endpoints.
The second, `uneven' observing strategy schedules the observations
according to
\begin{align}
t_i = \left(\frac{i - \frac{1}{2}}{n}\right)^3
\end{align}
which schedules observations close together initially, and further apart
as time goes on.

A purely sinusoidal signal has the form
\begin{align}
y(t) &= A \sin \left(\frac{2\pi t}{T} + \phi\right)
\end{align}
where $A$ is the amplitude, $T$ is the period, and $\phi$ is the phase.
Throughout this section, I parameterise the period by its logarithm,
$\tau = \log_{10} T$. I assumed the following priors:
\begin{align}
\ln A   &\sim \textnormal{Normal}\left(0, 0.1^2\right)  \\
\tau    &\sim \textnormal{Uniform}(-1, 0)  \\
\phi    &\sim \textnormal{Uniform}(0, 2\pi)
\end{align}
and the following conditional prior for the data:
\begin{align}
d_i | A, T, \phi &\sim \textnormal{Normal}\left(y(t_i), 0.1^2\right).
\end{align}
That is, the data is just the signal $y(t)$ observed at particular times
$\{t_1, ..., t_n\}$, with gaussian noise of standard deviation 0.1.
The amplitude of the sinusoid is very likely to be around 10 times the
noise level, and the period is between 0.1 and 1 times the duration of
the data. An example signal observed with the even and uneven observing schedules
is shown in Figure~\ref{fig:sinewave}.

\begin{figure}[!ht]
\centering
\includegraphics[width=0.7\linewidth]{figures/sinewave.pdf}
\caption{A signal with true parameters $A=1$, $\tau=-0.5$, and
$\phi=0$, observed with noise standard deviation 0.1 with the
even (gold points) and uneven (green points) observing strategies.
\label{fig:sinewave}}
\end{figure}


\subsection{Results}

Letting $\tau = \log_{10} T$, and treating
$A$ and $\phi$ are nuisance parameters, I calculated the conditional entropy
$H(\tau | \d)$ using the algorithm with a tolerance of $L = 10^{-5}$.
The result was
\begin{align}
\depth &= 5.379 \pm 0.038
\end{align}
Converting to a differential entropy by adding $\ln (2 \times 10^{-5})$, the conditional
entropy is
\begin{align}
H(\tau | \d) &= -5.441 \pm 0.038 \textnormal{ nats}.
\end{align}
Since the Uniform(0,1) prior has a differential entropy of zero, the
mutual information is $I(\tau; \d) = H(\tau) - H(\tau | \d) = 5.441 \pm 0.038$
nats.

The results for the uneven observing schedule were
\begin{align}
\depth       &=  5.422 \pm 0.038 \\
H(\tau | \d) &= -5.398 \pm 0.038 \\
I(\tau; \d)   &=  5.398 \pm 0.038 \textnormal{ nats}.
\end{align}

The difference in mutual informations between the two observation
schedules is trivial.
The situation may be different if we
had allowed for shorter periods to be possible, as irregular observing
schedules are known to reduce aliasing and ambiguity in the inferred period.
When inferring the period of an oscillating signal, multimodal posterior
distributions for the period are common \citep{gregoryTrimodal, exoplanet}.
The multimodality of the posterior here raises an interesting issue. Is
the question we really want answered ``what is the value of $T$
{\bf precisely}?'', to which the mutual information relates?
Most practicing scientists would not feel particularly informed to learn
that the vast majority of possibilities had been ruled out, if the
posterior still consisted of several widely separated modes!
Perhaps, in some applications, a more appropriate question is
``what is the value of $T$ to within $\pm$ 10\%'', or something along these
lines.

In a serious experimental design situation, the relevance is not the only
consideration (or the answer to every experimental design problem would
be to obtain more data without limit), but it is a very important one.

\subsection{Computational cost}

The computational resources needed to compute these quantities was quite large,
as Nested Sampling appears in the inner loop of the algorithm.
However, it is worth reflecting on the complexity of the calculation that
has been done.

In a Bayesian inference problem with parameters $\x$ and data $\d$, the
mutual information is
\begin{align}
I(\x; \d) &= \iint p(\x, \d)
                        \ln \left[\frac{p(\x, \d)}{p(\x)p(\d)}\right]
                        \, d\x \, d\d \label{eqn:mutual_info2}
\end{align}
and this measures the dependence between $\x$ and $\d$. A
Monte Carlo
strategy to evaluate this is to sample from $p(\x, \d)$ and average
the value of the logarithm, which involves computing the log-evidence
\begin{align}
\ln p(\d) = \ln\left[\int p(\x)p(\d | \x) \, d\x\right]
\end{align}
for each possible data set.
Computing $p(\d)$ for even a single
data set has historically been considered difficult.

However, if we want the mutual information between the data and
{\em a subset of the parameters} (i.e., there are nuisance parameters
we don't care about), things become even more tricky. Suppose we
partition the parameters $\x$ into important parameters
$\bphi$ and nuisance parameters $\boldeta$, such that we want to calculate
the mutual information $I(\bphi; \d)$. This is given by
\begin{align}
I(\bphi; \d) &= \iint p(\bphi, \d)
                        \ln \left[\frac{p(\bphi, \d)}{p(\bphi)p(\d)}\right]
                        \, d\bphi \, d\d \label{eqn:mutual_info3}
\end{align}
which is equivalent to Equation~\ref{eqn:mutual_info2} but with
$\x$ replaced throughout by $\bphi$. However, this is an even more
difficult calculation than before, as $p(\d | \bphi)$, the marginal likelihood
function with the nuisance parameters integrated out, is typically
unavailable in closed form. If we were to marginalise out the nuisance
parameters $\boldeta$ explicitly, this would give us
Equation~\ref{eqn:mutual_info3} with every probability distribution
written as an explicit integral over $\boldeta$:

\begin{align}
I(\bphi; \d) &= \iint
  \left(\int p(\bphi, \boldeta)p(\d | \bphi, \boldeta) \, d\boldeta\right)
                        \ln \left[
  \frac{\int p(\bphi, \boldeta)p(\d | \bphi, \boldeta) \, d\boldeta}
{p(\bphi)\int p(\bphi, \boldeta)p(\d | \bphi, \boldeta) \, d\bphi \, \d\boldeta}\right]
                        \, d\bphi \, d\d
\end{align}
It should not be surprising that this is costly.

In the sinusoidal signal example, I calculated the difference in
mutual information between two observing strategies. It is possible
that there exists a more direct method of obtaining this which
is more efficient than running the two strategies separately.

\section{Example 3: TBD}

\acknowledgments{Acknowledgements}
It is a pleasure to thank the following people for interesting and helpful
conversations about this topic: Ruth Angus (Flatiron Institute),
Ewan Cameron (Oxford), James Curran (Auckland),
David Hogg (NYU), Kevin Knuth (SUNY Albany),
Thomas Lumley (Auckland),
Iain Murray (Edinburgh), Jared Tobin ({\tt jtobin.io}).
This work was supported by Centre for eResearch
at the University of Auckland.

%%%%%%%%%%%%%%%%%%%%%%%%%%%%%%%%%%%%%%%%%%

%\authorcontributions{Author Contributions}

%%%%%%%%%%%%%%%%%%%%%%%%%%%

\conflictofinterests{Conflicts of Interest}
The authors declare no conflicts of interest.

%=================================================================
% References: Variant A
%=================================================================
% Back Matter (References and Notes)
%----------------------------------------------------------
% Style and layout of the references
%\bibliographystyle{mdpi}
\makeatletter
\renewcommand\@biblabel[1]{#1. }
\makeatother


%=================================================================
% References:  Variant B
%=================================================================
% Use the following option to include external BibTeX files:
\bibliographystyle{mdpi}
\bibliography{references}

\appendix
\section{Precisional questions}

The central issue about the value of a parameter $\x$ asks
``what is the value of $\x$, precisely?''. However, in practice we often
don't need or want to know $\x$ to arbitrary precision. Using the central
issue can lead to counterintuitive results if you don't keep its specific
definition in mind. For example, suppose $x$ could take any integer value
from 1 to 1 billion. If we learned the last digit of $x$, we
will have ruled out nine tenths of the possibilities, and therefore obtained
a lot of information about the central issue. However,
this information might be useless {\em for practical purposes}.
If we learned the final
digit was a 9, $x$ could still be 9, or 19, or 29, any number ending in 9
up to 999,999,999.

In practice, we may
want to ask a question that is different from the central issue.
For example, suppose $x \in \{1, ..., 10\}$ and
we want to know the value of $x$ to within a tolerance
of $\pm 1$. Any of the following statements would
resolve the issue:
\begin{itemize}
\item $x \in \{1, 2, 3\}$ and anything that implies it
\item $x \in \{2, 3, 4\}$ and anything that implies it
\item $x \in \{3, 4, 5\}$ and anything that implies it
\item and so on.
\end{itemize}
According to \citet{knuth_questions}, the entropy of a question is
computed by applying the sum rule over all statements that would answer the
question.

We can write the question of interest, $Q$, as a union of ideal questions:
\begin{align}
Q &= \left[\downarrow (x \in \{1, 2, 3\})\right] \cup
     \left[\downarrow (x \in \{2, 3, 4\})\right] \cup
     \left[\downarrow (x \in \{3, 4, 5\})\right] \cup ...
\end{align}
The entropy of this question is
\begin{align}
H(Q) &= h_{1,2,3}\\
     & \quad \quad + \left(h_{2,3,4} - h_{2,3}\right) \\
     & \quad \quad + \left(h_{3,4,5} - h_{3,4}\right) \\
     & \quad \quad + ... \\
     & \quad \quad + \left(h_{8,9,10} - h_{8,9}\right).
\end{align}
where $h_{x} = -P(x)\log P(x)$.

\subsection*{Continuous case}

Consider a probability density function $f(x)$, defined on the real
line. The probability contained in an interval
$[x_0, x_0 + L]$, which has length $L$, is
\begin{align}
\int_{x_0}^{x_0 + L} f(x) \, dx &= F(x_0 + L) - F(x_0)
\end{align}
where $F(x)$ is the cumulative distribution function (CDF).
This probability can be considered as a function of two variables,
$x_0$ and $L$, which I will denote by $P(,)$:
\begin{align}
P(x_0, L) &= F(x_0 + L) - F(x_0).
\end{align}
The contribution of such an interval to an entropy expression
is $-P\log P$, i.e.,
\begin{align}
Q(x_0, L) &= -P(x_0, L) \log P(x_0, L).
\end{align}
Consider the rate of change of $Q$ as the interval is shifted to
the right, but with its width $L$ held constant:
\begin{align}
\frac{\partial Q}{\partial x_0} &= \frac{\partial}{\partial x_0}
    \left[-P\log P\right] \\
    &= -\left[1 + \log P(x_0, L)\right]\frac{\partial P}{\partial x_0} \\
    &= -\left[1 + \log P(x_0, L)\right]\left[f(x_0 + L) - f(x_0)\right].
\end{align}
Consider also the rate of change of $Q$ as the interval is
expanded to the right, while keeping its left edge fixed:
\begin{align}
\frac{\partial Q}{\partial L} &= \frac{\partial}{\partial L}
    \left[-P(x_0, L) \log P(x_0, L)\right] \\
    &= -\left[1 + \log P(x_0, L)\right]\frac{\partial P}{\partial L} \\
    &= -\left[1 + \log P(x_0, L)\right]f(x_0 + L).
\end{align}

The entropy of the precisional question is built up from
$Q$ terms. The extra entropy from adding the interval
$[x_0, x_0 + L]$ and removing the overlap $[x_0, x_0 + L - h]$, for small $h$, is
\begin{align}
\delta H &= Q(x_0, L) - Q(x_0, L - h)\\
         &= h\frac{\partial Q}{\partial L} \\
         &= -h\left[1 + \log P(x_0, L)\right]f(x_0 + L).
\end{align}
Therefore, the overall entropy is
\begin{align}
H &= -\int_{-\infty}^\infty
        \left[1 + \log P(x-L, L)\right]f(x)
      \, dx \\
  &= -\int_{-\infty}^\infty
        \left(1 + \log\left[F(x) - F(x-L)\right]\right)f(x)
      \, dx.
\end{align}
Interestingly, calculating the log-probability of an interval
to the right of $x$ gives an equivalent result:
\begin{align}
H' &=  -\int_{-\infty}^\infty
        \left(1 + \log\left[F(x+L) - F(x)\right]\right)f(x)
      \, dx \\
   &= H.
\end{align}
This is not quite the same as the expected log-probability calculated by
the version of the algorithm proposed in this paper (when the tolerance is not
small). However, the algorithm can be made to estimate the entropy of the
precisional question by redefining the distance function such that
$d(\x; \xref)$ can only ever be below $L$ if $\x < \xref$ (or alternatively,
if $\x > \xref$).

\appendix
\section{Software}

A {\tt C++} implementation of the algorithm is available in a {\tt git}
repository located at
\begin{verbatim}
https://github.com/eggplantbren/InfoNest
\end{verbatim}
and can be obtained using the following {\tt git} command, executed in a
terminal:
\begin{verbatim}
git clone https://github.com/eggplantbren/InfoNest
\end{verbatim}
The following will compile the code and execute the first example from the
paper:
\begin{verbatim}
cd InfoNest/cpp
make
./main
\end{verbatim}
The algorithm will run for 1000 `reps', i.e., 1000 samples of $\xref$, which is
time consuming. Output is saved to {\tt output.txt}. At any time,
you can execute the Python script {\tt postprocess.py} to get an estimate of
the depth:
\begin{verbatim}
python postprocess.py
\end{verbatim}



%%%%%%%%%%%%%%%%%%%%%%%%%%%%%%%%%%%%%%%%%%

%\abbreviations{Abbreviations/Nomenclature}
%
%Main text.

%%%%%%%%%%%%%%%%%%%%%%%%%%%%%%%%%%%%%%%%%%


\end{document}

