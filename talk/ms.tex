\documentclass{beamer}
\usepackage[utf8]{inputenc}
\usepackage{palatino}
\usepackage{subfig}
\usepackage{amsmath}
\usepackage{dsfont}
\usepackage{multimedia}

\usetheme{Warsaw}
\usecolortheme{crane}

% www.sharelatex.com/learn/Beamer

\title{Computing Entropies with Nested Sampling}
\author{Brendon J. Brewer}
\institute{Department of Statistics\\
The University of Auckland}
\date{{\tt \color{blue} https://www.stat.auckland.ac.nz/\~{ }brewer/}}

\begin{document}

\frame{\titlepage}


% New slide
\begin{frame}
\frametitle{What is entropy?}

Firstly, I'm talking about information theory, not thermodynamics (though the
two are connected).

\end{frame}


% New slide
\begin{frame}
\frametitle{What is entropy?}

Firstly, I'm talking about information theory, not thermodynamics (though the
two are connected).

\end{frame}




% New slide
\begin{frame}
\frametitle{References I.}

On the connection between Shannon entropy and thermodynamic entropy,
see: \vspace{2em}

{\footnotesize Jaynes, Edwin T. ``Gibbs vs Boltzmann entropies.''
American Journal of Physics 33, no. 5 (1965): 391-398. \\
}

\end{frame}



% New slide
\begin{frame}
\frametitle{References II.}

On the connection between Shannon entropy and thermodynamic entropy,
see: \vspace{2em}

{\footnotesize Jaynes, Edwin T. ``Gibbs vs Boltzmann entropies.''
American Journal of Physics 33, no. 5 (1965): 391-398. \\

Brewer, Brendon J. ``Unscrambling the Second Law of Thermodynamics''
{\color{blue} \tt
http://quillette.com/2016/01/28/unscrambling-the-second-law-of-thermodynamics/}
}

\end{frame}



\end{document}


